\documentclass[onecolumn,12pt]{article}
\usepackage[utf8]{inputenc}
\usepackage{amsmath}
\author{Nelle Varoquaux}

\title{Finding Structure with Randomness: Probabilistic Algorithms for
Constructing Approximate Matrix Decompositions}
\begin{document}
\maketitle
\begin{abstract}
\it{}
\end{abstract}

\tableofcontents

\section{Introduction}

In many fields, decompositions are used to implement efficient matrix
algorithms. In many situations, classical algorithms have become inadequate:

\begin{itemize}
\item In information science, matrices can be extremely big.
\item New developments in software hardware add new constraints on the size of
matrices. It becomes increasingly important to be able to use graphics
processing units for performance.
\item Information can be missing or inaccurate in some fields of information
science. If classical algorithms yield highly accurate results, it may not be
necessary to spend resources on a high quality result, if the inherent
source limit the results.
\end{itemize}

We will study in this article how randomized algorithms provide strong tools
for constructing such approximation of matrix decomposition. We will focus on
the quality of the results obtained with such a decomposition, and leave the
effectiveness beside.

% FIXME TODO finish me

\subsection{Pseudo algorithm}
The task of computing a low rank approximation can be split into two stages.

The first task is to compute a low dimensional subspace that captures most of
the action of the matrix. The second is to restric the matrix on that subspace
and compute the decomposition.

\section{Stage A algorithms}

\section{Stage B algorithms}

\section{Applications}
\subsection{A large dense matrix}
This example involves a large dense matrix, derives from the Olivetti face
dataset, often use for facial recognition. One of the famous method to do such
task involves extracting principal direction, called \textbf{eigenfaces}. Each
image can then be summarized by its component along the axis of the principal
direction. A classifier such as a LinearSVM is then applied, and yields good
results.




\subsection{A large sparse matrix}

This example involves a large matrix that arises in image processing. Some
image processing algorithms uses the geometry of the image for tasks such as
denoising, inpainting \dots. They use a graph laplacian to represent the
geometry of the image.

We use with a small grayscaled patch of lena, of $100 x 100$. Each pixel is
represented by a value between $0$ and $255$. Each pixel $x$ is represented by
a patch of $6x6$ around this pixel. We can then form the weight matrix
$\tilde{W}$, reflecting the similarity between the patch with:

$$\tilde{w_{ij}} = exp \{ \frac{- (x_i - x_j)^2}{\sigma^2}\}$$

By zeroing all the entries of the weigth matrix $\tilde{W}$ except the seven
largest ones in earch row, we construct our large sparse matrix $W$.

We can then construct the graph Laplacian matrix:

$$L = I - D^{-1/2} W D^{1/2}$$.

% FIXME
We want to extract the eigenvectors of the matrix $A= D^{1/2} W D^{-1/2}$.In
order to do so, we use the Algorithm X

\end{document}
